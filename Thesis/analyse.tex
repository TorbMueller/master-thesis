\chapter{Analysis}
\label{cha:analyse}

Lorem ipsum dolor sit amet, consetetur sadipscing elitr, 
sed diam nonumy eirmod tempor invidunt ut labore et dolore
 magna aliquyam erat, sed diam voluptua. At vero eos et
  accusam et justo duo dolores et ea rebum. Stet clita kasd 
  gubergren, no sea takimata sanctus est Lorem ipsum dolor sit
   amet. Lorem ipsum dolor sit amet, consetetur sadipscing
    elitr, sed diam nonumy eirmod tempor invidunt ut labore
     et dolore magna aliquyam erat, sed diam voluptua. 
     At vero eos et accusam et justo duo dolores et ea rebum.
      Stet clita kasd gubergren, no sea takimata sanctus est
       Lorem ipsum dolor sit amet.
\section{Ausgangssituation}
Lorem ipsum dolor sit amet, consetetur sadipscing elitr, 
sed diam nonumy eirmod tempor invidunt ut labore et dolore
 magna aliquyam erat, sed diam voluptua. At vero eos et
  accusam et justo duo dolores et ea rebum. Stet clita kasd 
  gubergren, no sea takimata sanctus est Lorem ipsum dolor sit
   amet. Lorem ipsum dolor sit amet, consetetur sadipscing
    elitr, sed diam nonumy eirmod tempor invidunt ut labore
     et dolore magna aliquyam erat, sed diam voluptua. 
     At vero eos et accusam et justo duo dolores et ea rebum.
      Stet clita kasd gubergren, no sea takimata sanctus est
       Lorem ipsum dolor sit amet.



\subsection{Probleme der aktuellen Infrastruktur}
Die aktuelle Infrastruktur des hostings der Web-Apps zeigt mehrere signifikante Probleme, die eine Migration in eine moderne Cloud-Umgebung
erforderlich machen. Ein wesentliches Problem ist die begrenzte Skalierbarkeit der bestehenden Umgebung, die aus zwei dedizierten Servern besteht.
Diese Architektur stößt bei zunehmender Last schnell an ihre Grenzen, da eine horizontale Skalierung nur mit erheblichem Aufwand möglich ist.
Zusätzliche Server müssen manuell bereitgestellt und in die bestehende Umgebung integriert werden. Das führt zu einer eingeschränkten Flexibilität -
insbesondere bei plötzlichen Lastspitzen oder einem generellen Anstieg der Nutzung.

Ein weiterer kritischer Punkt ist der hohe Wartungsaufwand, der durch die manuelle Verwaltung der Server entsteht. Regelmäßige Software-Updates, und
die Überwachung der Systemressourcen erfordern umfassende Eingriffe und binden wertvolle personelle Ressourcen. Diese manuellen Prozesse erhöhen zudem
die Wahrscheinlichkeit menschlicher Fehler, was die Zuverlässigkeit und Stabilität der gesamten Infrastruktur beeinträchtigen kann.

Zusätzlich bietet die derzeitige Infrastruktur eine eingeschränkte Ausfallsicherheit. Da die Anwendungen ausschließlich auf zwei dedizierten Servern
betrieben werden, fehlt es an einer umfassenden Redundanz. Der Ausfall eines Servers könnte zu signifikanten Unterbrechungen führen, da es keine
integrierten Mechanismen gibt, die eine automatische Übernahme der Dienste ermöglichen. Besonders die produktive Umgebung ist dadurch potenziell
anfällig für Ausfallzeiten, was die Verfügbarkeit und Verlässlichkeit der Anwendungen gefährdet.

Ein weiteres Problem betrifft die Flexibilität bei Updates und Deployments. Obwohl die bestehende \ac{CI}/\ac{CD}-Pipeline mit GitHub Actions und
einer internen Ablage für Container-Images eine gewisse Automatisierung bietet, erfordert das manuelle Ausführen der Befehle ``docker compose pull''
und ``docker compose up -d'' nach wie vor menschliche Eingriffe auf den Servern. Dies kann zu Verzögerungen führen und erschwert schnelle und
konsistente Deployments, da der Prozess nicht vollständig automatisiert ist.

Die eingeschränkte Netzwerkintegration und die externe Erreichbarkeit stellen ebenfalls eine Herausforderung dar. Aktuell ist der Zugriff auf die
Web-Apps ausschließlich auf das interne Netzwerk beschränkt. Das würde bei einer notwendigen Erweiterung der Zugänglichkeit zu erheblichen Anpassungen
und potenziellen Sicherheitsrisiken führen. Die aktuelle Architektur ist nicht ausreichend darauf ausgelegt, die notwendigen Sicherheitsmaßnahmen zu
implementieren, um eine sichere und dennoch flexible externe Erreichbarkeit zu ermöglichen.

Zudem gibt es Sicherheitsbedenken, insbesondere in Bezug auf die Handhabung sensibler Konfigurationsdaten. Zwar werden die tatsächlichen Werte der
`.env-Dateien` nicht in das Git-Repository gepusht, um die Vertraulichkeit zu wahren, jedoch besteht bei unsachgemäßer Handhabung oder durch
menschliche Fehler weiterhin das Risiko einer versehentlichen Offenlegung. Die momentane Infrastruktur bietet nur begrenzte Unterstützung für
erweiterte Sicherheitsmaßnahmen, wie sie in modernen Cloud-Umgebungen Standard sind, etwa automatisierte Überwachungs- und Schutzmechanismen.

Ein abschließender Punkt betrifft die Kosten- und Ressourceneffizienz. Der Betrieb dedizierter Server verursacht konstante Kosten, unabhängig von der
tatsächlichen Nutzung der Ressourcen. Diese feste Kostenstruktur kann zu ineffizienten Ausgaben führen, da eine dynamische Anpassung der Ressourcen an
die tatsächliche Auslastung nicht möglich ist. Eine flexible On-Demand-Skalierung, wie sie in Cloud-Umgebungen Standard ist, kann die Effizienz und
Wirtschaftlichkeit der Ressourcen deutlich verbessern - ein Konzept, das die bestehende Infrastruktur nicht leisten kann.
\newpage
\subsection{Ziel der Migration}
\label{ssec:ziel_der_migration}

Das Hauptziel der Migration besteht darin, die bestehende Infrastruktur zu modernisieren und eine zukunftssichere, skalierbare, leistungsstarke sowie
kosteneffiziente Umgebung für das Web-Portal und die darin laufenden Anwendungen zu schaffen. Im Rahmen der Modernisierung sollen die folgenden
zentralen Herausforderungen adressiert werden:

\begin{itemize}
      \item \textbf{Verbesserung der Skalierbarkeit}: Die aktuelle Infrastruktur stößt insbesondere bei Lastspitzen schnell an ihre Grenzen. Durch die
            Migration in die Cloud soll eine flexible, bedarfsgesteuerte Skalierung ermöglicht werden, die eine schnelle Anpassung der Ressourcen an
            die aktuelle Nutzungslast erlaubt. Dies gewährleistet eine stabile und performante Benutzererfahrung, auch bei starkem Zugriff.

      \item \textbf{Erhöhung der Ausfallsicherheit und Verfügbarkeit}: Cloud-Umgebungen bieten umfangreiche Redundanz- und Hochverfügbarkeitslösungen
            \cite{8604034}\cite{Riaz2018}. Automatische Failover-Mechanismen sowie geografisch verteilte Rechenzentren minimieren potenzielle
            Ausfallzeiten und ermöglichen den kontinuierlichen Betrieb auch bei Serverausfällen oder Störungen \cite{Riaz2018}.

      \item \textbf{Automatisierung und Effizienzsteigerung}: Durch die Verwendung von Konzepten wie GitOps kann der Deployment-Prozess umfassend
            automatisiert werden. Dies reduziert den manuellen Aufwand, minimiert menschliche Fehler und beschleunigt die Bereitstellung neuer
            Funktionen \cite{9565152}.

      \item \textbf{Steigerung der Kosteneffizienz}: Im Gegensatz zu den festen Kosten für dedizierte Server bietet die Cloud flexible Abrechnungsmodelle
            basierend auf der tatsächlichen Nutzung (Pay-as-you-go). Die automatische Anpassung der Ressourcen an den Bedarf verhindert
            Überkapazitäten und steigert somit die Kosteneffizienz, da nur bezahlt wird, was auch genutzt wird. \cite{6009283}

      \item \textbf{Verbesserung der Erreichbarkeit}: Während die bestehende Infrastruktur auf internen Zugriff beschränkt ist, ermöglicht die Cloud
            eine externe Erreichbarkeit der Anwendungen.
\end{itemize}

Zusammenfassend soll die Migration die bestehenden Einschränkungen der aktuellen Infrastruktur überwinden und eine moderne, flexible und sichere
Umgebung schaffen, die den wachsenden Anforderungen des Unternehmens gerecht wird.

\section{Anforderungen an die zukünftige Infrastruktur}
\label{sec:anforderungsanalyse}
% Wichtig !!! Anforderungsanalyse nach folgendem Buch: https://www.sophist.de/publikationen/requirements-engineering-und-management/ LIV:
% https://bsz.ibs-bw.de/aDISWeb/app;jsessionid=DE11A9B8B156B28F0E43F3B61CC56A91 Webseite zu Buch:
% https://www.hanser-elibrary.com/doi/book/10.3139/9783446464308 PDF: https://www.hanser-elibrary.com/doi/epdf/10.3139/9783446464308

Das folgende Kapitel orientiert sich am Verfahren von Rupp\cite{Rupp2021}. Im Rahmen dieser Arbeit wird bei der Anforderungsanalyse gemäß
\cite[Kap.~16.3.1]{Rupp2021} eine natürlichsprachliche Anforderungsspezifikation gewählt, um eine bessere Verständlichkeit zu gewährleisten. Zur
Darstellung der Anforderungen kommt die Use-Case-basierte Analyse gemäß \cite[Kap.~18.2.1]{Rupp2021} zum Einsatz. Diese Methode erfasst klare
Anwendungsfälle in einem ablauforientierten Kontext und ermöglicht eine strukturierte Darstellung der relevanten Prozesse im Migrationsprojekt.

Die Anforderungsanalyse dient dazu, die funktionalen Anforderungen, wie zum Beispiel die spezifischen Skalierbarkeits- oder Sicherheitsanforderungen
der neuen Infrastruktur, sowie die nicht-funktionalen Anforderungen, wie Verfügbarkeit und Performanz, systematisch zu erfassen und zu dokumentieren.
Sie bildet damit die Grundlage für die Konzeption der Zielarchitektur und deren Implementierungsschritte im Migrationsprozess.

Die einzelnen Anforderungen werden dabei nach der in \cite[Kap.~19.3]{Rupp2021} beschriebenen Schablone erstellt, um eine konsistente und einheitliche
Dokumentation sicherzustellen. Diese Schablone definiert eine standardisierte Struktur, die alle relevanten Aspekte einer Anforderung - wie die
Bedingungen, das betroffene System, die Funktionalität und das Zielobjekt - präzise und nachvollziehbar beschreibt. Durch die Nutzung der Schablone
wird sichergestellt, dass alle Anforderungen vollständig und ohne Missverständnisse formuliert werden.

Eine ausführliche Darstellung der verwendeten Schablone, einschließlich ihrer Bestandteile und einer Beispielstruktur, findet sich im Anhang (siehe
Abschnitt~\ref{template}). Diese Schablone dient als Grundlage, um die Anforderungen konsistent zu dokumentieren und eine klare Kommunikation
innerhalb des Projektes zu fördern. Die strukturierte Herangehensweise erleichtert zudem die spätere Verifikation und Validierung der Anforderungen
sowie ihre Umsetzung im Rahmen der Migration.

Die Anforderungen an die Cloud-Migration werden in drei zentrale Kategorien unterteilt: technische, organisatorische und regulatorische Anforderungen.
Diese Differenzierung ermöglicht es, die spezifischen Herausforderungen aus unterschiedlichen Perspektiven zu betrachten und gezielt darauf
einzugehen. Technische Anforderungen betreffen dabei Aspekte wie Skalierbarkeit, Performance und Sicherheit der Infrastruktur, während
organisatorische Anforderungen die internen Prozesse und Zuständigkeiten abdecken. Regulatorische Anforderungen beziehen sich auf die Einhaltung
gesetzlicher Vorgaben und Richtlinien.

Die Anforderungsanalyse ist darauf ausgerichtet, diese Kategorien systematisch zu erfassen und zu dokumentieren, um sicherzustellen, dass keine
wesentlichen Aspekte übersehen werden. Sie bildet die Grundlage für die Konzeption der zukünftigen Infrastruktur und gewährleistet, dass alle
Anforderungen nachvollziehbar und präzise in den späteren Entwicklungs- und Implementierungsphasen berücksichtigt werden.

Die Erhebung und Formulierung der folgenden Anforderungen erfolgt in enger Zusammenarbeit mit der Robert Bosch Manufacturing Solutions GmbH, um
sicherzustellen, dass diese den spezifischen Erwartungen und betrieblichen Anforderungen des Unternehmens präzise entsprechen.

\subsection{Technische Anforderungen}

Die technischen Anforderungen beschreiben die essenziellen Merkmale und Fähigkeiten, die das zukünftige System in der Cloud erfüllen muss, um eine
sichere, flexible und leistungsfähige Infrastruktur bereitzustellen. Sie bilden die Grundlage für den Aufbau und die Konfiguration der
Cloud-Infrastruktur und legen fest, wie zentrale Aspekte wie Skalierbarkeit, Redundanz und Sicherheitsstandards umgesetzt werden sollen. Eine
detaillierte Auflistung dieser Anforderungen erfolgt im nächsten Abschnitt.

\begin{itemize}
      \item \hypertarget{Anforderung1.1}{\textbf{Anforderung 1.1: Selbstständige Skalierung bei Lastspitzen}}

            Sobald entweder der Speicher, oder die Recheneinheit vollständig ausgelastet ist, muss das System fähig sein, automatisch zusätzliche
            Ressourcen bereitzustellen.
            \newpage
      \item \hypertarget{Anforderung1.2}{\textbf{Anforderung 1.2: Redundante Datenverfügbarkeit}}

            Solange das System in Betrieb ist, muss das System fähig sein, Daten in geografisch verteilten Rechenzentren zu speichern.

      \item \hypertarget{Anforderung1.3}{\textbf{Anforderung 1.3: Verfügbarkeit}}

            Solange das System in Betrieb ist, muss das System fähig sein, sowohl von innerhalb als auch von außerhalb des Unternehmensnetzwerks verfügbar
            zu sein.

      \item \hypertarget{Anforderung1.4}{\textbf{Anforderung 1.4: Automatisierte Updates}}

            Solange das System in Betrieb ist, muss das System fähig sein, Updates selbstständig und ohne spürbare Unterbrechung der Dienste
            einzuspielen.

      \item \hypertarget{Anforderung1.5}{\textbf{Anforderung 1.5: \gls{loadbalancing}}}

            Solange das System in Betrieb ist, muss das System fähig sein, den eingehenden Netzwerkverkehr durch einen Lastverteiler auf mehrere Instanzen der
            Anwendung zu verteilen.

      \item \hypertarget{Anforderung1.6}{\textbf{Anforderung 1.6: Datensicherung}}

            Solange das System in Betrieb ist, muss das System fähig sein, in regelmäßigen Abständen automatische Backups der gesamten Datenbanken zu
            erstellen und diese in einem sicheren Speicher abzulegen.

      \item \hypertarget{Anforderung1.7}{\textbf{Anforderung 1.7: Verschlüsselung}}

            Sobald eine Anfrage eintritt, muss das System fähig sein, Daten bei der Übertragung zu verschlüsseln.

      \item \hypertarget{Anforderung1.8}{\textbf{Anforderung 1.8: Anbieterunabhängigkeit}}

            Sobald das System in Betrieb genommen wird, sollte die Infrastruktur dem Inbetriebnehmer die Möglichkeit bieten den Cloud-Anbieter, ohne
            weitere Anpassungen vornehmen zu müssen, unter den führenden Anbietern frei auswählen zu können.
            \newpage
      \item \hypertarget{Anforderung1.9}{\textbf{Anforderung 1.9: Regionalität}}

            Solange das System in Betrieb ist, muss das System den Administratoren die Möglichkeit geben, Einstiegspunkte in verschiedenen global
            verteilten Regionen zu schaffen.
\end{itemize}


\subsection{Organisatorische Anforderungen}

Die organisatorischen Anforderungen umfassen interne Richtlinien und Sicherheitsanforderungen, die den Migrationsprozess in die Cloud maßgeblich
beeinflussen. Sie stellen sicher, dass alle unternehmensinternen Vorgaben sowie regulatorischen und sicherheitsrelevanten Bestimmungen eingehalten
werden, um einen konformen und sicheren Betrieb der Anwendungen in der Cloud zu gewährleisten. Im Folgenden werden die spezifischen organisatorischen
Anforderungen aufgeführt, die bei der Planung und Umsetzung der Migration berücksichtigt werden müssen.

\begin{itemize}
      \item \hypertarget{Anforderung2.1}{\textbf{Anforderung 2.1: Zugriffskontrolle}}

            Solange das System in Betrieb ist, muss das System fähig sein, auf Grundlage eines rollenbasierten Zugriffskontrollmodells Berechtigungen zu
            verwalten. \cite{BoschCentralDirectiveAccessManagement}\footnote{Quelle aus dem Intranet (nicht öffentlich zugänglich) von BOSCH.}

      \item \hypertarget{Anforderung2.2}{\textbf{Anforderung 2.2: Zugriff auf Protokolldaten nur für autorisierte Personen}}

            Solange Protokolldaten vorliegen, muss das System fähig sein, diese nur für autorisierte Personen zugänglich zu machen.

      \item \hypertarget{Anforderung2.3}{\textbf{Anforderung 2.3: Benachrichtigung bei sicherheitsrelevanten Ereignissen}}

            Sobald ein sicherheitsrelevantes Ereignis eintritt, muss das System fähig sein, eine Benachrichtigung an die zuständigen Personen zu senden.

      \item \hypertarget{Anforderung2.4}{\textbf{Anforderung 2.4: Unternehmensrichtlinien}}

            Solange Cloud-Dienste genutzt werden, muss das System fähig sein, alle geltenden internen Unternehmensrichtlinien für die Cloud-Nutzung
            nach \cite{BoschCompanyGuidelinesCloud}\footnote{Quelle aus dem Intranet (nicht öffentlich zugänglich) von BOSCH.} einzuhalten.

      \item \hypertarget{Anforderung2.5}{\textbf{Anforderung 2.5: Erfahrung}}

            Bereits zu Beginn der Migration muss das System so ausgewählt werden, dass im Unternehmen umfassendes Know-how in der verwendeten Technologie
            vorhanden ist.

      \item \hypertarget{Anforderung2.6}{\textbf{Anforderung 2.6: Integration}}

            Sobald die zukünftige Infrastruktur auf der neuen Cloud-basierten Lösung aufgebaut wird, sollte die zukünftige Infrastruktur fähig sein, sich
            bestmöglich in die bestehenden Systeme und Prozesse zu integrieren.
\end{itemize}


\subsection{Regulatorische Anforderungen}
Die regulatorischen Anforderungen umfassen Datenschutzbestimmungen, Compliance-Vorgaben und weitere gesetzliche Richtlinien, die für den sicheren und
rechtskonformen Betrieb der Anwendungen in der Cloud erforderlich sind. Diese Anforderungen stellen sicher, dass alle rechtlichen Verpflichtungen
eingehalten werden und das Unternehmen in Übereinstimmung mit geltenden Vorschriften und Standards agiert. Im Folgenden sind die spezifischen
regulatorischen Anforderungen aufgeführt, die bei der Planung und Umsetzung der Migration berücksichtigt werden.

\begin{itemize}
      \item \hypertarget{Anforderung3.1}{\textbf{Anforderung 3.1: Datenschutz}}

            Solange personenbezogene Daten verarbeitet werden, muss das System fähig sein, alle Anforderungen der \ac{DSGVO} zu erfüllen.

      \item \hypertarget{Anforderung3.2}{\textbf{Anforderung 3.2: Zertifizierte Rechenzentren}}

            Solange Daten gespeichert werden, muss das System sicherstellen, dass diese ausschließlich in Rechenzentren mit gültigen Zertifizierungen
            gemäß internationalen Sicherheitsstandards abgelegt werden.
            \newpage
      \item \hypertarget{Anforderung3.3}{\textbf{Anforderung 3.3: Datenverarbeitung auf Servern innerhalb der EU}}

            Solange personenbezogene Daten von Benutzern innerhalb der EU verarbeitet werden, muss das System sicherstellen, dass diese nur auf Servern
            innerhalb der Europäischen Union verarbeitet werden, sofern keine explizite Einwilligung der betroffenen Personen zur Verarbeitung außerhalb
            der EU vorliegt.

      \item \hypertarget{Anforderung3.4}{\textbf{Anforderung 3.4: Transparenz}}

            Solange das System in Betrieb ist, muss das System fähig sein, zu jeder Zeit alle Protokolldaten vollständig und nachvollziehbar zu erfassen
            und transparent bereitzustellen.
\end{itemize}

\section{Evaluierung der Cloud-Anbieter}
\label{sec:evaluation_cloud_provider}

Die Auswahl eines geeigneten Cloudanbieters stellt einen zentralen Schritt im Prozess der Migration der Web-Applikationen dar und hat erhebliche
Auswirkungen auf die Erreichbarkeit, Skalierbarkeit und Kosteneffizienz der Infrastruktur. Ziel dieser Evaluierung ist es daher, die in Betracht
kommenden Cloudanbieter hinsichtlich ihrer Dienste, Kostenstrukturen und ihrer Fähigkeit, die definierten Anforderungen zu erfüllen, systematisch zu
vergleichen.

Im Rahmen der Evaluierung werden ausschließlich die aktuell führenden Cloudanbieter in Betracht gezogen. Diese sind \ac{AWS}, Azure und Google Cloud
Platform \cite{Borra2024}. Durch die Fokussierung auf diese Marktführer wird sichergestellt, dass die Untersuchung auf hoch relevante und
praxiserprobte Technologien abzielt, die auch in anspruchsvollen Anwendungsszenarien verlässliche Ergebnisse liefern können.

In den folgenden Abschnitten werden zunächst die wichtigsten Dienste der führenden Cloudanbieter vergleichend dargestellt. Anschließend erfolgt eine
detaillierte Kostenanalyse, bevor abschließend die Anbieter anhand der zuvor definierten Anforderungen evaluiert werden.

\subsection{Dienste}
In diesem Abschnitt werden die zentralen Dienste der relevanten Cloudanbieter im Kontext der geplanten Cloud-Migration analysiert und dargestellt. Die
Struktur der dazugehörigen Tabelle orientiert sich an der Arbeit von \cite{Borra2024} und \cite{Solangi2023}. \footnote{Einträge, die nicht direkt aus
      einer der genannten Quellen entnommen sind, stammen aus eigenständiger Recherche auf den offiziellen Plattformen der Anbieter, einschließlich Tools
      wie dem Preiskalkulationsprogramm.}

Die Auswahl der in der Tabelle aufgeführten Kategorien basiert auf den spezifischen Anforderungen der migrierenden Infrastruktur. Sie umfassen
Dienste, die essenziell für den Betrieb und die Wartung der cloudbasierten Anwendungen sind:

\begin{itemize}
      \item \textbf{Verwaltung des Lebenszyklus}: Diese Kategorie umfasst Tools, die den Entwicklungs- und Bereitstellungsprozess unterstützen, um
            kontinuierliche Integrationen und automatisierte Deployments zu ermöglichen.
      \item \textbf{Kubernetes}: Kubernetes-Dienste sind zentral für das Management der Anwendungen in einem Kubernetes Cluster.
      \item \textbf{Datenbanken}: Hier werden sowohl relationale Datenbanken wie PostgreSQL als auch NoSQL-Datenbanken wie MongoDB betrachtet, da
            diese zur Speicherung und Verwaltung der Anwendungsdaten verwendet werden.
      \item \textbf{Überwachung}: In dieser Kategorie sind Monitoring-Lösungen aufgeführt, die zur Überwachung verschiedener Systemparameter dienen.
            Sie ermöglichen einen tieferen Einblick in das System und das Monitoring kritischer Metriken.
\end{itemize}

Um die Übersichtlichkeit zu gewährleisten, wurden alle Dienste, die für die in dieser Arbeit untersuchte Cloud-Migration keine Bedeutung besitzen,
nicht mit einbezogen. So werden beispielsweise Dienste im Zusammenhang mit maschinellem Lernen nicht berücksichtigt, da im zu migrierenden Portal und
den darin enthaltenen Applikationen gegenwärtig keine künstliche Intelligenz eingesetzt wird.

\renewcommand{\arraystretch}{1.2}
\newcolumntype{X}[1]{>{\raggedright\arraybackslash}p{#1}}

\begin{table}[H]
      \centering
      \caption{Vergleich der Cloud-Dienste führender Anbieter}
      \label{tab:cloud_services_comparison}
      \begin{tabular}{|X{3cm}|X{4cm}|X{4cm}|X{4cm}|}
            \hline
            \renewcommand*\acfsfont{\textbf} \textbf{Kategorie} & \textbf{\ac{AWS}}                  & \textbf{Microsoft Azure}      & \textbf{Google
            Cloud Platform}                                                                                                                                                 \\
            \hline
            Kubernetes                                          & Amazon Elastic Kubernetes Services & \ac{AKS}                      & Google Kubernetes Engine             \\
            \hline
            MongoDB-Datenbanken                                 & Amazon DocumentDB                  & Cosmos DB                     & Firestore                            \\
            \hline
            PostgreSQL                                          & Amazon RDS for PostgreSQL          & Azure Database for PostgreSQL & Cloud SQL for PostgreSQL             \\
            \hline
            Überwachung                                         & Amazon Managed Grafana             & Azure Managed Grafana         & Grafana Cloud observability platform \\
            \hline
      \end{tabular}
\end{table}

Aus der Tabelle wird ersichtlich, dass alle betrachteten Cloudanbieter die benötigten Dienste bereitstellen.

\subsection{Kostenvergleich}

In diesem Abschnitt werden die Kosten für die Konfiguration der Produktivumgebung der betrachteten Migrationskandidaten (siehe Abschnitt
\ref{sec:evaluation_cloud_provider}) miteinander verglichen. Die Produktivumgebung ist für die Analyse von besonderer Bedeutung, da die Testumgebung
auf derselben Infrastruktur betrieben werden soll. Eine separate Kostenaufstellung für diese ist daher nicht notwendig.

Bei der Konfiguration der \ac{GCP} müssen einige Annahmen getroffen werden, da spezifische Angaben für die Firestore-Datenbank, wie etwa die Anzahl
der IO-Operationen (Schreibe- und Leseoperationen) oder Request Units, nicht allgemein konfigurierbar sind und daher einzeln angegeben werden müssen.
Um eine Vergleichbarkeit zu gewährleisten, wird angenommen, dass 2/3 der Firestore-Operationen Leseoperationen und 1/3 Schreiboperationen ausmachen.
Diese Aufteilung basiert auf einer Kombination aus typischen Zugriffsmustern \cite{10.14778/3401960.3401967}, die häufig mehr Lese- als
Schreibzugriffe aufweisen, sowie den spezifischen Erwartungen der Robert Bosch Manufacturing Solutions GmbH.

Ein weiterer Faktor, der die Kosten bei \ac{GCP} stark beeinflusst, ist die begrenzte Konfigurierbarkeit der Firestore-Datenbank\footnote{Information
      entstammt dem Google Preiskalkulationsprogramm}. Im Vergleich zu \ac{AWS} und Microsoft Azure bietet \ac{GCP} hier nur eingeschränkte
Möglichkeiten, Firestore-Instanzen individuell anzupassen. Diese mangelnde Flexibilität führt dazu, dass die Kosten für die Firestore Instanz im
Rahmen der Produktivumgebung unverhältnismäßig hoch ausfallen.

Die Kostenaufstellung im Anhang (siehe Abschnitt \ref{sec:kostenaufstellung_cloudanbieter}) zeigt deutliche Unterschiede in den monatlichen
Gesamtkosten der drei Anbieter. Die Kosten für \ac{AWS} belaufen sich auf \$1595,65 pro Monat, während Microsoft Azure mit \$1436,06 pro Monat leicht
günstiger ist. \ac{GCP} hingegen weist mit \$3400,86 pro Monat die mit Abstand höchsten Kosten auf. Diese Differenz lässt sich vor allem auf die zuvor
genannten Einschränkungen bei der Konfiguration von Firestore zurückführen.

Zusammenfassend zeigt der Vergleich, dass sowohl \ac{AWS} als auch Microsoft Azure kosteneffizienter für die hier betrachtete Produktivumgebung sind.
\ac{GCP} ist aufgrund der fehlenden Konfigurationsmöglichkeiten und der damit verbundenen hohen Kosten in diesem Szenario weniger geeignet und wird
daher nicht weiter untersucht.

Nach \cite{Rani2014} entsprechen Azure als auch \ac{AWS}, dem PaaS-Service-Modell. Die Wahl zwischen \ac{AWS} und Microsoft Azure hängt also letztlich
von der Evaluation der Anforderungen in Sektion \ref{ssec:requirements_evaluation_cloud_provider} ab.

\subsection{Evaluation nach Anforderungen}
\label{ssec:requirements_evaluation_cloud_provider}

Aufgrund der in der Kostenaufstellung (siehe Abschnitt \ref{sec:kostenaufstellung_cloudanbieter}) identifizierten hohen monatlichen Kosten der Google
Cloud Platform wird diese im weiteren Verlauf nicht berücksichtigt, da sie für die betrachtete Produktivumgebung als weniger geeignet erscheint. Die
Evaluation konzentriert sich daher auf die beiden wirtschaftlich sinnvollen Anbieter \ac{AWS} und Microsoft Azure. Um eine fundierte Entscheidung
zwischen diesen Anbietern zu treffen, erfolgt ein Vergleich basierend auf den in Sektion \ref{sec:anforderungsanalyse} definierten Anforderungen.

Dabei liegt der Fokus auf den Anforderungen \hyperlink{Anforderung1.2}{1.2}, \hyperlink{Anforderung1.3}{1.3}, \hyperlink{Anforderung1.6}{1.6},
\hyperlink{Anforderung1.7}{1.7}, \hyperlink{Anforderung1.9}{1.9}, \hyperlink{Anforderung2.1}{2.1}, \hyperlink{Anforderung2.2}{2.2},
\hyperlink{Anforderung2.3}{2.3}, \hyperlink{Anforderung2.5}{2.5}, \hyperlink{Anforderung2.6}{2.6}, \hyperlink{Anforderung3.1}{3.1},
\hyperlink{Anforderung3.2}{3.2}, \hyperlink{Anforderung3.3}{3.3} und \hyperlink{Anforderung3.4}{3.4}. Diese Anforderungen wurden ausgewählt, da ihre
Erfüllung direkt von den jeweiligen Angeboten der Cloudanbieter abhängt und nicht von der spezifischen Implementierung oder Konfiguration der
Infrastruktur.

Das Ziel dieses Kapitels ist es, die Anbieter \ac{AWS} und Azure daraufhin zu vergleichen, wie gut ihre Dienste die definierten Anforderungen
erfüllen. Dies schafft eine fundierte Basis für die Auswahl des passenden Cloudanbieters.

\subsubsection{Methodik}

Die Evaluation der Cloudanbieter basiert auf der Analyse ihrer bereitgestellten Dienste, Sicherheitsmechanismen und Infrastrukturmaßnahmen. Hierfür
werden öffentlich zugängliche Dokumentationen, Fachartikel sowie Zertifizierungen nach internationalen Standards herangezogen.

Zur Bewertung wird ein qualitativer Bewertungsmaßstab verwendet, da viele Anforderungen nicht direkt messbar sind oder stark vom spezifischen Szenario
abhängen. Beispielsweise erfordert die Bewertung der Integration in bestehende Systeme oder der Einhaltung von Datenschutzbestimmungen eine
kontextspezifische Analyse, die sich nicht sinnvoll in Zahlen darstellen lässt. Ein weiterer Vorteil dieses Ansatzes liegt in der Transparenz: Durch
die verbale Beschreibung der Erfüllung wird nachvollziehbar dargestellt, warum ein Anbieter bestimmte Anforderungen erfüllt oder nicht. Dies
verhindert, dass komplexe Sachverhalte vereinfacht und potenzielle Unterschiede zwischen den Anbietern übersehen werden.

Die Erfüllung der Anforderungen wird dabei wie folgt kategorisiert:
\begin{itemize}
      \item \textbf{Erfüllt}: Der Anbieter erfüllt die Anforderung vollständig, ohne erkennbare Einschränkungen.
      \item \textbf{Nicht erfüllt}: Der Anbieter erfüllt die Anforderung nicht oder bietet keine entsprechenden Mechanismen an.
\end{itemize}

Diese Kategorisierung gewährleistet, basierend auf den Anforderungen, eine objektive und nachvollziehbare Bewertung, während die qualitative
Einordnung eine praxisnahe Entscheidungsgrundlage für die Auswahl des geeigneten Cloudanbieters bietet.

\subsubsection{Beurteilung}
\label{ssec:beurteilung}
In diesem Abschnitt werden die zuvor definierten Anforderungen auf ihre Erfüllung durch die beiden verbleibenden Cloudanbieter, \ac{AWS} und Microsoft
Azure, untersucht.

Um eine fundierte Entscheidungsgrundlage zu schaffen, welche Plattform die betrachteten Anforderungen besser erfüllt, werden die Stärken und Schwächen
beider Anbieter analysiert, um die jeweiligen Vor- und Nachteile herauszuarbeiten.

Die Beurteilung wird zunächst in Form von erklärenden Texten zu den einzelnen Anforderungen durchgeführt. Die logische Unterteilung der aufgezählten
Anforderungen entspricht der aus Kapitel \ref{sec:anforderungsanalyse}. Anschließend folgt eine tabellarische Aufstellung der Ergebnisse, um die
Erfüllung der Anforderungen durch die beiden Anbieter übersichtlich und kompakt darzustellen.

\begin{itemize}
      \item \textbf{Technische Anforderungen}:
            \begin{itemize}
                  \item \textbf{\hyperlink{Anforderung1.2}{Anforderung 1.2 (Redundante Datenverfügbarkeit)}}: \ac{AWS} und Microsoft Azure bieten beide
                        Mechanismen zur redundanten Datenverfügbarkeit an, die den Anforderungen entsprechen. \ac{AWS} stellt durch seine
                        Multi-AZ-Bereitstellungen sicher, dass Daten in mehreren geografisch verteilten Rechenzentren gespeichert werden können.
                        \cite{AWS-Multi-AZ-SQL} \cite{AWS-Multi-AZ-MongoDB} Ähnliches bietet Azure mit der Unterstützung von Zonenredundanz, die ebenfalls
                        eine geografisch verteilte Speicherung der Daten ermöglicht. \cite{Azure-redundancy-SQL}
                        \cite{Azure-redundancy-MongoDB}
                  \item \textbf{\hyperlink{Anforderung1.3}{Anforderung 1.3 (Verfügbarkeit)}}: Beide Anbieter garantieren eine hohe Verfügbarkeit ihrer
                        Dienste. \ac{AWS} gibt ein Service Level Agreement, je nach gebuchter Option von mindestens 99\% für seine Hauptdienste an, während
                        Microsoft ähnliche Werte für seine Azure Services bietet. \cite{AWS-Availability} \cite{Azure-Availability}
                  \item \textbf{\hyperlink{Anforderung1.6}{Anforderung 1.6 (Datensicherung)}}: \ac{AWS} bietet automatische Backups über seine Dienste
                        wie Amazon RDS und \ac{AWS} Backup, die regelmäßige Sicherungen unterstützen und diese in einem sicheren Speicher ablegen, an.
                        \cite{AWS-Backup-RDS}\cite{AWS-Backup-Service} Azure bietet, mit in den Datenbanksystemen eingebauten Funktionen,
                        \cite{Perry2021}\cite{AzureMongoDbAutomaticBackups} Mechanismen an, die ebenfalls regelmäßige Datensicherungen und sichere
                        Speicherorte gewährleisten.
                        \newpage
                  \item \textbf{\hyperlink{Anforderung1.7}{Anforderung 1.7 (Verschlüsselung)}}: \ac{AWS} und Azure unterstützen beide die
                        Verschlüsselung von Daten während der Übertragung und im Ruhezustand. Beide Anbieter bieten durch Standards wie \ac{TLS} und
                        AES-256 sichere Verschlüsselungsmethoden an, die den Anforderungen entsprechen. \cite{AWS-Encryption-Rest}
                        \cite{AWS-Encryption-Transit}
                  \item \textbf{\hyperlink{Anforderung1.9}{Anforderung 1.9 (Regionalität)}}: \ac{AWS} und Azure bieten beide die Möglichkeit,
                        Einstiegspunkte in global verteilten Regionen flexibel zu schaffen. \ac{AWS} verfügt über ein Netzwerk an Rechenzentren mit
                        mehreren Verfügbarkeitszonen in zahlreichen Ländern weltweit, was eine schnelle Bereitstellung in verschiedenen Regionen
                        ermöglicht. Microsoft Azure bietet ebenfalls ein weitreichendes globales Netzwerk mit 54 Regionen auf allen Kontinenten und
                        orientiert sich dabei gezielt an regionalen Marktanforderungen und regulatorischen Vorgaben. \cite{Bögelsack2022}
            \end{itemize}
      \item \textbf{Organisatorische Anforderungen}:
            \begin{itemize}
                  \item \textbf{\hyperlink{Anforderung2.1}{Anforderung 2.1 (Zugriffskontrolle)}}: \ac{AWS} verwendet sein sogenanntes Identity and Access
                        Management, um eine rollenbasierte Zugriffskontrolle zu ermöglichen. Azure bietet mit Azure Entra ID ebenfalls ein leistungsfähiges
                        Tool zur Verwaltung von Berechtigungen und Zugriffen an. \cite{AWS-IAM-Roles} \cite{Azure-EntraId-Roles}
                  \item \textbf{\hyperlink{Anforderung2.2}{Anforderung 2.2 (Zugriff auf Protokolldaten nur für autorisierte Personen)}}: Durch die in
                        der Bewertung der vorhergehenden Anforderung erwähnten Services, ist es bei beiden Anbietern möglich Protokolldaten vor
                        unbefugtem Zugriff zu schützen. Dies resultiert daraus, dass der Zugriff auf Applikationen in beiden Fällen beschränkt werden
                        kann. Aufgrund dessen gilt diese Anforderung ebenfalls von beiden Cloudanbietern als erfüllt. \cite{AWS-IAM-Roles}
                        \cite{Azure-EntraId-Roles}
                        \newpage
                  \item \textbf{\hyperlink{Anforderung2.3}{Anforderung 2.3 (Benachrichtigung bei sicherheitsrelevanten Ereignissen)}}: \ac{AWS} und
                        Microsoft Azure bieten mit Amazon GuardDuty bzw. Azure Defender for Cloud umfassende Lösungen, um sicherheitsrelevante
                        Ereignisse zu erkennen und entsprechende Benachrichtigungen auszulösen. Beide Dienste ermöglichen es Administratoren,
                        automatisierte Benachrichtigungen und Reaktionen auf sicherheitskritische Vorfälle zu konfigurieren. Dabei integrieren sie
                        sich nahtlos in die jeweilige Sicherheitsinfrastruktur der Anbieter und bieten umfangreiche Anpassungsmöglichkeiten.
                        \cite{AWS-GuardDuty} \cite{AWS-GuardDuty-Notifications} \cite{Azure-DefenderForCloud}
                        \cite{Azure-DefenderForCloud-Notifications}
                  \item \textbf{\hyperlink{Anforderung2.5}{Anforderung 2.5 (Erfahrung)}}: \ac{AWS} und Azure sind beides bekannte Anbieter mit breiter
                        Unterstützung in der Industrie. Dennoch ist das im Unternehmen bereits verfügbare Know-how über Azure größer. Dies liegt
                        daran, dass es zum Zeitpunkt der Erstellung dieser Arbeit mehr interne Dokumentationen über Azure gibt als über \ac{AWS}. So
                        gibt es laut der Bosch-internen Suchmaschine für Dokumentationen circa 28.553 Beiträge über die \ac{AWS} Cloud und rund 60.080
                        Artikel über die Azure Cloud im unternehmensinternen Archiv. Somit gibt es mehr als doppelt so viele interne Quellen über
                        Azure, als über \ac{AWS}. Dies führt dazu, dass die \hyperlink{Anforderung2.5}{Anforderung 2.5} für Microsoft Azure als
                        erfüllt, aber für \ac{AWS} als nicht erfüllt gilt.
                        % Herausgefunden indem "Azure Cloud" und "AWS Cloud" in der Suchmaschine eingegeben wurde und die Anzahl Ergebnisse angeschaut wurde
                        % (23.11.2024)
                        \newpage
                  \item \textbf{\hyperlink{Anforderung2.6}{Anforderung 2.6 (Integration)}}: Die Integration in bestehende Systeme und Prozesse spielt
                        eine entscheidende Rolle bei der Wahl des Cloudanbieters. Microsoft Azure bietet hier einen deutlichen Vorteil, da bereits
                        eine wesentliche Komponente aus dem Azure-Ökosystem unternehmensweit genutzt wird. Azure Entra ID (ehemals Azure Active
                        Directory) dient als zentrale Lösung für die Verwaltung von Benutzern im gesamten Unternehmen
                        \cite{BoschEntraId}\footnote{Quelle aus dem Intranet (nicht öffentlich zugänglich) von BOSCH.}. Dadurch ist eine nahtlose
                        Integration der Azure-Cloud-Dienste in die bestehenden Systeme gewährleistet, ohne dass zusätzliche Anpassungen erforderlich
                        sind.

                        Im Gegensatz dazu wird von \ac{AWS} aktuell keine Technologie unternehmensweit verwendet. Im Fall einer Nutzung von \ac{AWS}
                        müsste beispielsweise ebenfalls Azure Entra ID für die Authentifizierung eingebunden werden, da dies die zentrale
                        Benutzerverwaltung im Unternehmen darstellt \cite{BoschEntraId}\footnote{Quelle aus dem Intranet (nicht öffentlich zugänglich)
                              von BOSCH.}. Das bedeutet, dass eine Integration von \ac{AWS} eine Kombination aus beiden Plattformen erfordert, was zu
                        erhöhtem Implementierungsaufwand führen könnte. Dadurch ergibt sich ein Vorteil für eine rein auf Azure basierende Lösung, da
                        hier keine zusätzlichen Schnittstellen oder Anpassungen notwendig sind.

            \end{itemize}
      \item \textbf{Regulatorische Anforderungen}:
            \begin{itemize}
                  \item \textbf{\hyperlink{Anforderung3.1}{Anforderung 3.1 (Datenschutz)}}: Die Verarbeitung personenbezogener Daten unterliegt in Deutschland
                        den strengen Vorgaben der \ac{DSGVO}. Sowohl \ac{AWS} als auch Microsoft Azure stellen laut den Angaben auf ihren Webseiten sicher,
                        dass ihre Dienste alle Anforderungen der \ac{DSGVO} erfüllen. \cite{AWS-Datenschutz} \cite{Azure-Datenschutz}

                  \item \textbf{\hyperlink{Anforderung3.2}{Anforderung 3.2 (Zertifizierte Rechenzentren)}}: Die Nutzung zertifizierter Rechenzentren ist
                        essenziell, um die Sicherheit und Verfügbarkeit von Daten gemäß internationalen Standards zu gewährleisten. Sowohl \ac{AWS} als auch
                        Microsoft Azure betreiben Rechenzentren, die nach der umfassenden ISO 27001 Norm zertifiziert sind. Diese stellt sicher, dass die
                        Anbieter strikte Maßnahmen zur Informationssicherheit implementieren, einschließlich Zugriffskontrollen, Risikomanagement und Schutz
                        der Datenintegrität.

                        Darüber hinaus verfügen beide Anbieter über weitere relevante Zertifizierungen, wie beispielsweise ISO 27018 für den Schutz
                        personenbezogener Daten in der Cloud. \cite{AWS-Datenschutz} \cite{Azure-Datenschutz}

                  \item \textbf{\hyperlink{Anforderung3.3}{Anforderung 3.3 (Datenverarbeitung auf Servern in der EU)}}: \ac{AWS} und Azure stellen
                        beide sicher, dass personenbezogene Daten auf Servern innerhalb der EU verarbeitet werden können. So konnte mithilfe des
                        Preiskalkulationsprogramms der jeweiligen Anbieter herausgefunden werden, dass \ac{AWS} dies unter anderem über seine
                        Rechenzentren in Frankfurt und Irland ermöglicht, während Azure über seine europäischen Regionen ebenfalls eine Verarbeitung
                        innerhalb der EU gewährleistet.

                  \item \textbf{\hyperlink{Anforderung3.4}{Anforderung 3.4 (Transparenz)}}: Die Transparenz in der Protokollierung und Bereitstellung von
                        Protokolldaten ist essenziell, um die Nachvollziehbarkeit und Sicherheit der Systeme zu gewährleisten. Sowohl \ac{AWS} als auch
                        Microsoft Azure bieten umfassende Lösungen für die transparente Erfassung und Bereitstellung von Protokolldaten.

                        \ac{AWS} stellt Dienste wie \ac{AWS} CloudTrail zur Verfügung, mit denen alle \ac{API}-Aufrufe und Aktionen innerhalb der
                        Infrastruktur detailliert protokolliert werden. Diese Logs sind vollständig durchsuchbar und lassen sich exportieren, um sie
                        in andere Systeme zu integrieren. Azure bietet mit Azure Monitor ähnliche Funktionen, die Protokolldaten in Echtzeit erfassen
                        und eine umfassende Nachvollziehbarkeit gewährleisten. Beide Anbieter ermöglichen es, Protokolldaten über sichere Kanäle an
                        autorisierte Personen weiterzugeben. \cite{AWS-CloudTrail} \cite{AWS-CloudTrail-Export} \cite{Azure-Monitor}
                        \cite{Azure-Monitor-Overview}
            \end{itemize}
\end{itemize}

\renewcommand{\arraystretch}{1.5}

\begin{longtable}{|p{8.5cm}|c|c|}
      \caption{Erfüllung der Anforderungen durch AWS und Microsoft Azure}
      \label{tab:anforderungen_erfüllung}                                                                                                  \\
      \hline
      \textbf{Anforderung}                                                                      & \textbf{AWS}  & \textbf{Microsoft Azure} \\ \hline
      \endfirsthead
      \hline
      \textbf{Anforderung}                                                                      & \textbf{AWS}  & \textbf{Microsoft Azure} \\ \hline
      \endhead
      \hline
      \multicolumn{3}{r}{\textit{Fortsetzung auf der nächsten Seite}}                                                                      \\ \hline
      \endfoot
      \hline
      \endlastfoot
      \hyperlink{Anforderung1.2}{1.2}: Redundante Datenverfügbarkeit                            & Erfüllt       & Erfüllt                  \\ \hline
      \hyperlink{Anforderung1.3}{1.3}: Verfügbarkeit                                            & Erfüllt       & Erfüllt                  \\ \hline
      \hyperlink{Anforderung1.6}{1.6}: Datensicherung                                           & Erfüllt       & Erfüllt                  \\ \hline
      \hyperlink{Anforderung1.7}{1.7}: Verschlüsselung                                          & Erfüllt       & Erfüllt                  \\ \hline
      \hyperlink{Anforderung1.9}{1.9}: Regionalität                                             & Erfüllt       & Erfüllt                  \\ \hline
      \hyperlink{Anforderung2.1}{2.1}: Zugriffskontrolle                                        & Erfüllt       & Erfüllt                  \\ \hline
      \hyperlink{Anforderung2.2}{2.2}: Zugriff auf Protokolldaten nur für autorisierte Personen & Erfüllt       & Erfüllt                  \\ \hline
      \hyperlink{Anforderung2.3}{2.3}: Benachrichtigung bei sicherheitsrelevanten Ereignissen   & Erfüllt       & Erfüllt                  \\ \hline
      \hyperlink{Anforderung2.5}{2.5}: Erfahrung                                                & Nicht erfüllt & Erfüllt                  \\ \hline
      \hyperlink{Anforderung2.6}{2.6}: Integration                                              & Nicht erfüllt & Erfüllt                  \\ \hline
      \hyperlink{Anforderung3.1}{3.1}: \ac{DSGVO}-Konformität                                   & Erfüllt       & Erfüllt                  \\ \hline
      \hyperlink{Anforderung3.2}{3.2}: Zertifizierte Rechenzentren                              & Erfüllt       & Erfüllt                  \\ \hline
      \hyperlink{Anforderung3.3}{3.3}: Datenverarbeitung auf Servern innerhalb der EU           & Erfüllt       & Erfüllt                  \\ \hline
      \hyperlink{Anforderung3.4}{3.4}: Transparenz                                              & Erfüllt       & Erfüllt                  \\ \hline
\end{longtable}

Die Tabelle zeigt, dass sowohl \ac{AWS} als auch Microsoft Azure die meisten der definierten Anforderungen erfüllen. Besonders in den Bereichen
Verfügbarkeit, Sicherheit und Datenschutz sind beide Anbieter gut aufgestellt und erfüllen die Anforderungen vollständig. Dies verdeutlicht, dass
beide Plattformen für eine produktive Nutzung im Unternehmen grundsätzlich geeignet sind.

Jedoch gibt es Unterschiede in der Erfüllung spezifischer Anforderungen. Während \ac{AWS} bei der Integration und Erfahrung Defizite aufweist, erfüllt
Azure diese Anforderungen vollständig. Dies ist insbesondere darauf zurückzuführen, dass die Azure-Technologie, Entra ID zur Verwaltung der
Zugriffsrechte bereits im Unternehmen eingesetzt werden und sich somit auf bestehende Infrastruktur zurückgegriffen werden kann. \ac{AWS} hingegen ist
im aktuellen Unternehmenskontext nicht in die Systeme eingebunden, was die Integration erschwert.

Die vollständige Erfüllung der Anforderungen durch beide Anbieter in sicherheitskritischen Bereichen wie den Zertifizierten Rechenzentren und
Transparenz unterstreicht die hohe Qualität der Sicherheitsstandards bei beiden Anbietern. Beide Plattformen gewährleisten zudem die Einhaltung der
\ac{DSGVO}, was für die Verarbeitung personenbezogener Daten unerlässlich ist.

Insgesamt zeigt die Auswertung, dass Microsoft Azure aufgrund der besseren Integration in die bestehende Infrastruktur und der höheren vorhandenen
Erfahrung im Unternehmen in dieser Evaluation im Vorteil ist. Aus diesem Grund wird Azure als Cloudanbieter für die durchzuführende Migration gewählt.

\section{Risiken und Herausforderungen der Migration}
Die Migration der Anwendungen und Daten in die Cloud stellt einen komplexen Prozess dar, der mit einer Vielzahl von Risiken und Herausforderungen
verbunden ist. Neben technischen Aspekten wie der Anpassung bestehender Systeme und der Datenmigration spielen auch organisatorische und
wirtschaftliche Faktoren eine zentrale Rolle. Eine frühzeitige Identifikation und Analyse dieser Risiken ist entscheidend, um eine strukturierte
Planung und erfolgreiche Umsetzung der Migration zu gewährleisten.

In diesem Abschnitt werden die potenziellen Herausforderungen und Risiken betrachtet. Dabei liegt der Fokus auf technischen, organisatorischen,
sicherheitsrelevanten und wirtschaftlichen Aspekten, die für die Migration besonders kritisch sind. Ziel ist es, ein umfassendes Verständnis für die
möglichen Problemfelder zu vermitteln und Strategien zur Minimierung dieser Risiken aufzuzeigen. Dies soll als Grundlage für die erfolgreiche Planung
und Durchführung der Migration dienen.

\subsection{Technische Herausforderungen}
Die Migration in die Cloud bringt eine Vielzahl technischer Herausforderungen mit sich, die sorgfältig berücksichtigt werden müssen. Diese betreffen
insbesondere die Datenmigration sowie die Integration von Drittanbieterdiensten.

\subsubsection{Umzug der Infrastruktur}
Ein zentraler Aspekt der Migration ist die korrekte Konfiguration von Netzwerk- und Maschinenressourcen in der neuen Cloud-Umgebung. Die bestehende
Infrastruktur muss so abgebildet werden, dass sie den Anforderungen an Verfügbarkeit, Performance und Sicherheit entspricht. Dies umfasst die
Einrichtung virtueller Netzwerke, \glsfirst{loadbalancing} und Firewalls sowie die Definition von Zugriffskontrollen. Fehler bei der Konfiguration
können zu Leistungseinbußen oder Sicherheitsrisiken führen, weshalb dieser Prozess mit großer Sorgfalt durchgeführt werden muss.

Ein weiterer wichtiger Faktor ist die Skalierbarkeit der Infrastruktur, wie in Kapitel \ref{ssec:ziel_der_migration} beschrieben. Die Cloud ermöglicht
es grundsätzlich, Ressourcen dynamisch an wechselnde Anforderungen anzupassen. Damit Lastspitzen effektiv abgefangen und unnötige Kosten vermieden
werden, ist die Implementierung automatischer Skalierungsmechanismen wie der Container-Orchestrierung durch Kubernetes entscheidend. Nur durch eine
optimierte Integration dieser Technologie kann die Skalierung effizient und bedarfsgerecht erfolgen.

\subsubsection{Datenmigration}
Die Migration von Datenbanken in die Cloud stellt einen wichtigen Schritt im Migrationsprozess dar. Obwohl die Datenbanken in diesem Fall nicht so
groß sind, dass physische Transferlösungen wie Azure Data Box erforderlich wären, bleibt die Sicherstellung von Datenintegrität und Sicherheit während
des Transfers eine zentrale Herausforderung. Es muss gewährleistet sein, dass keine Daten verloren gehen und die übertragenen Daten korrekt in der
neuen Umgebung ankommen, ohne dass ein dritter die Möglichkeit hat mitlesen zu können. Sensible Daten müssen aufgrund dessen während der Übertragung
verschlüsselt werden, um unbefugten Zugriff zu verhindern.

\subsection{Organisatorische Herausforderungen}
Die organisatorischen Herausforderungen der Migration konzentrieren sich insbesondere auf die Bereiche Koordination und Planung sowie den Umgang mit
fehlender Vorerfahrung bei bestimmten Technologien.

\subsubsection{Koordination und Planung}
Eine erfolgreiche Migration in die Cloud erfordert eine sorgfältige Planung und präzise Koordination aller beteiligten. Dies schließt die Festlegung
klarer Zuständigkeiten, realistischer Zeitpläne und ausreichender Ressourcen ein. Die Komplexität der Migration macht es notwendig, dass alle Schritte
und Meilensteine frühzeitig definiert und dokumentiert werden, um Missverständnisse oder Verzögerungen zu vermeiden. Besonders kritisch ist die
Abstimmung zwischen den Teams, die für die Infrastruktur, die Datenmigration und die Implementierung der neuen Umgebung verantwortlich sind. Eine
mangelhafte Koordination kann zu ineffizientem Ressourceneinsatz und erhöhtem Arbeitsaufwand führen.

\subsubsection{Mangelnde Vorerfahrung mit Kubernetes}
Ein weiterer organisatorischer Aspekt betrifft den Umgang mit fehlender Vorerfahrung bei der Nutzung von Kubernetes, der zentralen Technologie für das
Management von Containern und die Orchestrierung der Cloud-Infrastruktur. Die gesamte Abteilung hat bisher keine praktische Erfahrung mit Kubernetes,
was die Einführung dieser Technologie verlangsamen kann. Dieser Umstand erfordert eine intensive Einarbeitung in die grundlegenden Konzepte und
Funktionen, um sicherzustellen, dass die Infrastruktur korrekt implementiert und effizient verwaltet werden kann.

\subsection{Sicherheitsrisiken}
Die Migration in die Cloud birgt verschiedene Sicherheitsrisiken, die sowohl während des Transfers als auch nach der Einrichtung der neuen Umgebung
auftreten können. Ein zentraler Aspekt ist die Gefahr von Datenlecks oder unbefugtem Zugriff während des Migrationsprozesses. Sensible Daten könnten
durch unsichere Verbindungen abgefangen werden, falls nicht ausreichend Schutzmaßnahmen wie Verschlüsselung implementiert werden.
\cite{DBLP:journals/corr/Hababeh15}

Nach der Migration besteht das Risiko von Fehlkonfigurationen in der neuen Cloud-Umgebung. Beispielsweise könnten falsche Einstellungen dazu führen,
dass Daten unbefugt eingesehen oder verändert werden können. \cite{DBLP:journals/corr/Hababeh15}

Zusätzlich ist, im Gegensatz zur ursprünglichen Infrastruktur, welche nur aus dem firmeninternen Netzwerk erreichbar war, mit Angriffen auf die
Cloud-Umgebung durch bösartige Akteure (zum Beispiel durch DDoS- oder Brute-Force-Angriffe), die die Verfügbarkeit und Sicherheit der Systeme
gefährden, zu rechnen. \cite{9606751}

\subsection{Wirtschaftliche Risiken}
Die Migration in die Cloud ist nicht nur technisch und organisatorisch anspruchsvoll, sondern birgt auch wirtschaftliche Risiken, die sorgfältig
berücksichtigt werden müssen. Diese betreffen vor allem unerwartete Kostenüberschreitungen und die langfristige Abhängigkeit von einem einzelnen
Cloudanbieter.

\subsubsection{Kostenüberschreitungen}
Ein potenzielles wirtschaftliches Risiko in Bezug auf die Cloud-Migration sind unvorhergesehene Kosten, die während des Prozesses entstehen können.
Dies umfasst zusätzliche Aufwände für die Konfiguration, die Anpassung bestehender Anwendungen und die Bewältigung technischer Herausforderungen. Auch
können nicht vollständig kalkulierte Gebühren für Cloud-Dienste wie Datenübertragung, Storage oder Skalierung die geplanten Budgets überschreiten.
Diese Risiken erfordern eine detaillierte Planung der Migration, um Überraschungen und Budgetüberschreitungen zu vermeiden.

\subsubsection{Abhängigkeit von einem Anbieter}
Ein bedeutendes wirtschaftliches Risiko bei der Nutzung von Cloud-Diensten ist die Abhängigkeit von einem einzelnen Anbieter. Diese entsteht
insbesondere dann, wenn die Infrastruktur vollständig auf die spezifischen Technologien oder proprietären Schnittstellen des Anbieters abgestimmt ist.
In solchen Fällen kann ein zukünftiger Wechsel zu einem anderen Anbieter mit erheblichem Aufwand und hohen Kosten verbunden sein. Ein solcher Wechsel
könnte dann notwendig sein, wenn der Cloudanbieter, bei welchem das System läuft, seine Preise so signifikant erhöht, dass ein anderer Anbieter
deutlich günstiger ist.

\section{Strategien zur Risikominimierung}
Um die in den vorherigen Abschnitten identifizierten Risiken zu minimieren, werden im Folgenden Strategien vorgestellt, die gezielt auf die
technischen, organisatorischen, sicherheitsrelevanten und wirtschaftlichen Herausforderungen eingehen. Diese Maßnahmen sind darauf ausgerichtet,
potenzielle Störungen zu vermeiden und einen reibungslosen Übergang zur Cloud sicherzustellen.

\subsection{Sicherstellung von Datenintegrität und Sicherheit}
Während der Datenmigration ist der Schutz sensibler Informationen von zentraler Bedeutung. Um unbefugten Zugriff zu verhindern, sollten sichere
Übertragungsprotokolle wie \ac{TLS} eingesetzt werden. Darüber hinaus sollte die Verschlüsselung aller Daten, sowohl während der Übertragung als auch
im Ruhezustand, gewährleistet werden. Azure verschlüsselt laut der offiziellen Dokumentation standardmäßig alle Daten im Ruhezustand und die
Verschlüsselung der Daten während der Übertragung im Betrieb wird durch die Verwendung von HTTPS gewährleistet. \cite{Azure-Encryption-Transit-Rest}

\subsection{Standardisierung zur Reduzierung der Abhängigkeit von einem Anbieter}
Um die Abhängigkeit von einem einzigen Cloudanbieter zu reduzieren, sollte die Erstellung der Infrastruktur möglichst unabhängig von proprietären
Technologien gestaltet werden. Dies kann durch Werkzeuge wie Terraform, die es ermöglichen eine Infrastruktur deskriptiv zu definieren, verwirklicht
werden. Eine solche Herangehensweise ermöglicht es, die Infrastruktur bei Bedarf einfacher auf eine andere Plattform zu übertragen, was langfristig
die Flexibilität erhöht und das Risiko einer Bindung an einen Anbieter verringert. \cite{Howard2022}

\subsection{Umgang mit fehlender Vorerfahrung}
Da die gesamte Abteilung keine Vorerfahrung mit dem Einsatz von Kubernetes hat, wird die Einführung dieser Technologie durch den gezielten Einsatz
interner Ressourcen unterstützt. Eine Absprache mit der Robert Bosch Manufacturing Solutions GmbH stellt sicher, dass auf relevante interne
Dokumentationen sowie auf die Expertise erfahrener Mitarbeiter aus anderen Abteilungen zugegriffen werden kann. Dies fördert den Wissenstransfer und
ermöglicht eine effektive Einarbeitung der Abteilung in die neue Technologie.

\subsection{Sorgfältige Planung und Koordination}
Eine detaillierte Planung aller Migrationsschritte ist essenziell, um technische und organisatorische Herausforderungen zu bewältigen. Dazu gehört
unter anderem die klare Kommunikation der Projektziele an alle Beteiligten \cite{Cabarkapa2019/12}. Es kommen zudem regelmäßige Meetings und
Fortschrittsberichte zum Einsatz, um Missverständnisse so gut wie möglich zu vermeiden. Dadurch können potenzielle Probleme frühzeitig erkannt und
Verzögerungen aus dem Weg gegangen werden\footnote{Anforderung des Betreuers}.